\documentclass[xcolor=svgnames, usepdftitle=false, aspectratio=169]{beamer}
\usepackage{pgfpages}
\setbeameroption{hide notes}

\frenchspacing

% --- THEME ---
\usetheme[progressbar=frametitle]{metropolis}

\useoutertheme{metropolis}
\useinnertheme{metropolis}
\usefonttheme{metropolis}
\usecolortheme{dove}  % spruce, metropolis, dove, crane, beaver, seagull
\setbeamercolor{background canvas}{bg=transparent}
\usecolortheme[named=DarkOrange]{structure}
\usefonttheme[onlymath]{serif}

% \setbeamersize{text margin left=5.03cm}
\setbeamercolor{title separator}{fg=DarkOrange}

% --- HACKS ---
% Hide numbers on standout slides.
\setbeamertemplate{frame numbering}{
    \ifbool{metropolis@standout}{}{
        \insertframenumber
    }
}
\setbeamertemplate{frame numbering}[counter]  % none, counter, fraction

% Avoid font-warning with itemize bullets.
\renewcommand\textbullet{\ensuremath{\bullet}}

% --- PACKAGES ---
\usepackage[UKenglish]{babel}
\usepackage[utf8]{inputenc}
\usepackage{lmodern}
\usepackage[T1]{fontenc}

% \usepackage{appendixnumberbeamer}
\usepackage{upquote}
\usepackage[straightquotes]{newtxtt}
\usetikzlibrary{positioning}
% \usepackage{minted}
\usepackage{multicol}
\usepackage{xspace}
\usepackage{booktabs}
\usepackage{siunitx}

% --- SETTINGS ---
\graphicspath{{../figures/}}
\setlength{\fboxsep}{0pt}

% --- OWN COMMANDS ---
\newcommand{\bdra}{\ensuremath{\boldsymbol \Rightarrow }~}
\newcommand{\bdla}{\ensuremath{\boldsymbol \Leftarrow }~}
\newcommand{\dra}{\ensuremath{\Rightarrow }~}
\newcommand{\dla}{\ensuremath{\Leftarrow }~}
\newcommand{\mr}[1]{\mathrm{#1}}
\newcommand{\emg}[2]{\texttt{emg#1#2}\xspace}
\newcommand{\empymod}{\texttt{empymod}\xspace}
\newcommand{\ohmm}{\ensuremath{\Omega\,}\text{m}\xspace}
\newcommand{\rmk}[1]{{\color{red}\bfseries #1}}
\newcommand{\maybe}[1]{{\color{gray} #1}}
\newcommand{\todo}{{\color{red}\texttt{TODO:}}\xspace}
\newcommand{\bm}[1]{{\mathbf{#1}}}

% --- TITLE-STUFF ---

\newcommand{\ttitle}{Time-domain CSEM modelling}
\title{\vspace{2.5cm}\color{white}{\ttitle}}
\subtitle{\color{white}{using frequency- and Laplace-domain computations}}
\date{\color{white}{20 October 2021}}
\author{\vspace{-.3cm}\color{white}{Dieter Werthmüller and Evert Slob, TU Delft}}
\institute{}

\hypersetup{pdftitle={\ttitle}, allcolors=NavyBlue, colorlinks=true}

% --- SLIDES ---
\begin{document}
\metroset{block=fill}  % Fills the block-environment
\usebackgroundtemplate{\includegraphics[width=\paperwidth]{SlideTitle}}

\maketitle % ---------------------------------------------------------------- %
\usebackgroundtemplate{\includegraphics[width=\paperwidth]{SlideContent}}

\begin{frame}
  {Fast Fourier transform of EM data for computationally expensive kernels}

  \begin{itemize}
    \item Check and cite paper
    \item 15--25 frequencies
    \item Adaptive gridding; fct of skin depth
    \item Minimize frequencies; interpolation
    \item FFTLog or DLF
    \item Extrapolate, interpolate, set to zero
  \end{itemize}

\end{frame}

\begin{frame}
  {Example using FFTLog and Digital Linear Filters}
  \centering

  \bdra \quad (1) Frequency selection \quad (2) Gridding \quad \bdla\\[.5cm]

  \includegraphics[width=.8\textwidth]{fullspace}


  {\raggedright\small
  Werthmüller et al., 2021, \emph{Fast Fourier transform of electromagnetic
  data\\for computationally expensive kernels}, GJI; DOI:
  \href{https://doi.org/10.1093/gji/ggab171}{10.1093/gji/ggab171}.\\
  }


\end{frame}


\begin{frame}
  {Example: Induced Polarization}

  Also works for 3D or any model, as the transform is unaware of the
  complexity.

\end{frame}


\begin{frame}  % Motivation
  {Laplace-domain computation: Motivation}
  \centering
  \vspace{-.5cm}

  % Correct for: e^+iwt: + s σ E − ∇ × µ^-1 ∇ × E = − s J
  %              e^-iwt: + s σ E + ∇ × µ^-1 ∇ × E = − s J
  $$
    \mr{i}\omega \rightarrow s:
    \qquad
    \textcolor{red}{s} \sigma \mathbf{E} \ +\
    \nabla \times \mu^{-1} \nabla \times \mathbf{E}
    \ =\ -\textcolor{red}{s} \mathbf{J}_\mathrm{s}
  $$

  \bdra Faster\quad (1) Computation\quad (2) Convergence \quad \bdla\\
  \vspace{.5cm}

  \includegraphics[width=\textwidth]{motivationcomparison}

\end{frame}

\begin{frame}  % Speed & Convergence
  {Computation in layered and 3D codes}

  \begin{itemize}
    \item 30\,\% speed up in computation \& convergence (?)
    \item Computation speed-up: 1D \& 3D; convergence speed-up only 3D
    \item Works for semi-analytical responses, but not otherwise
  \end{itemize}
\end{frame}

\begin{frame}  % Filter design
  {Digital linear filter for Laplace}
\end{frame}


\begin{frame}
  {1D example: works!}

  \begin{enumerate}
    \item Design digital linear filters for the transform
    \item Carry out transform for semi-analytical (layered) responses
    \item Test stability
  \end{enumerate}


  \includegraphics[width=\linewidth]{s-t_time}%

\end{frame}

\begin{frame}
  {1D example: it doesn't!}

  Only works for \emph{very} precise results.

\end{frame}


\begin{frame}
  {Laplace-to-frequency domain}

  Idea and filter

\end{frame}

\begin{frame}
  {s-f}

  Example
\end{frame}


\begin{frame}
  {Conclusions}:

  Note (\todo delete): 15 Minutes max \dra 15 slides max

  Conclusions
  \begin{itemize}
    \item f\dra t: 15-25 frequencies
    \item Laplace: xx speedup for computation and convergence
    \item s\dra t; s\dra f: Only works for very precise results.
  \end{itemize}
\end{frame}

\begin{frame}%
  {References}
  \begin{columns}
    \column{\textwidth}
  \setlength{\columnseprule}{0.4pt}
  % \setlength{\columnsep}{3em}
  \begin{multicols}{2}
  \tiny
  \begin{description}[2cm]
    %
    \item[Ghosh, D. P., 1971,] {\bfseries The application of linear filter
      theory to the direct interpretation of geoelectrical resistivity sounding
      measurements:} Geophysical Prospecting, 19, 192--217;
      \href{https://doi.org/10.1111/j.1365-2478.1971.tb00593.x}{doi:~10.1111/j.1365-2478.1971.tb00593.x}.
    %
    \item[Hamilton, A. J. S., 2000,] {\bfseries Uncorrelated modes of the
      non-linear power spectrum:} Monthly Notices of the Royal Astronomical
      Society, 312, pages 257--284;
      \href{https://doi.org/10.1046/j.1365-8711.2000.03071.x}%
      {doi:~10.1046/j.1365-8711.2000.03071.x}.
    %
    \item[Mulder, W. A., M. Wirianto, and E. Slob, 2008,] {\bfseries
      Time-domain modeling of electromagnetic diffusion with a
      frequency-domain code:} Geophysics, 73, F1--F8;
      \href{https://doi.org/10.1190/1.2799093}{doi:~10.1190/1.2799093}.
    %
    \item[Plessix, R.-E., M. Darnet, and W. A. Mulder, 2007,] {\bfseries An
      approach for 3D multisource, multifrequency CSEM modeling:} Geophysics,
      72, SM177--SM184;
      \href{https://doi.org/10.1190/1.2744234}{doi:~10.1190/1.2744234}.
    %
    \item[Werthmüller, D., 2017,] {\bfseries An open-source full 3D
        electromagnetic modeler for 1D VTI media in Python: empymod:}
        Geophysics, 82(6), WB9--WB19;
        \href{https://doi.org/10.1190/geo2016-0626.1}%
        {doi:~10.1190/geo2016-0626.1}.
    %
    \item[Werthmüller, D., K. Key, and E. C. Slob, 2019,] {\bfseries A tool
        for designing digital filters for the Hankel and Fourier transforms
        in potential, diffusive, and wavefield modeling:} Geophysics, 84(2),
        F47--F56; \href{https://doi.org/10.1190/geo2018-0069.1}%
        {doi:~10.1190/geo2018-0069.1}.
    %
    \item[Werthmüller, D., W. A. Mulder, and E. C. Slob, 2019,] {\bfseries
      emg3d: A multigrid solver for 3D electromagnetic diffusion:} Journal of
      Open Source Software, 4(39), 1463;
      \href{https://doi.org/10.21105/joss.01463}{doi:~10.21105/joss.01463}.
    %
    \item[Werthmüller, D., W. A. Mulder, and E. C. Slob, 2021,] {\bfseries Fast
      Fourier transform of electromagnetic data for computationally expensive
      kernels:} Geophysical Journal International, 226, No. 2, 1336--1347;
      \href{https://doi.org/10.1093/gji/ggab171}{doi:~10.1093/gji/ggab171}.
    %
  \end{description}
\end{multicols}
\end{columns}~\\[.1cm]


Used open-source codes: \empymod (layered models) \& \emg3d (3D models),
\href{https://emsig.xyz}{emsig.xyz}.

Acknowledgment:\\
\scriptsize
This research was conducted within the Gitaro.JIM project funded\\
through MarTERA as part of Horizon 2020 (ERA-NET Cofund).



\end{frame}

\end{document}
